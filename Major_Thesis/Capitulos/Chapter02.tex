%*****************************************
\chapter{\textbf{Model and Samples}}\label{ch: Model}
%*****************************************

%============================================================================================================================================================
\section{Introduction}

%============================================================================================================================================================

%============================================================================================================================================================
\section{Power Law Distributions}\label{sec:PowerLawSec}

It is well-known that the stellar mass, and planetary mass and period can be described by power laws. As those parameters are essential in our formulation for the probability of exo-rings transits, and the subsequent analysis, we must pay attention to how model samples which can reproduce faithfully the observation.\\

A power law distribution can be defined as the relative change of two quantities, which are related through a common exponent. In other words, we can predict the change in one of the variables once the exponent and an initial set of values for the second variable are known. Mathematically speaking, we define a power law distribution as \autoref{eq:PowerLaw}, where $N$ and $X$ are the variables, and $\alpha$ is the exponent relating the relative change between them and it is assumed $\alpha \neq 1$. Generally, the variable $N$ refers to the number of objects one would expect to find in a given interval $x_1 \leq X \leq x_2$, where the variable $X$ in our particular case may refer to the planetary period, planetary mass, stellar mass or any other parameter which we want to study.\\ 

\begingroup
\Large
\begin{equation}
 \frac{\textnormal{dN}}{\textnormal{dX}}~~\propto~~\textnormal{X}^{-\alpha}
 \label{eq:PowerLaw}
\end{equation}
\endgroup\\

Power-laws can consist of a single or multiple exponents relating two or more variables. The easiest case is the single power law which has the mathematical form shown in \autoref{eq:PowerLaw}. However, the equation lacks of a proportionality constant or normalization constant which must be found with boundary conditions. Therefore, \autoref{eq:PowerLaw} can be rewritten in a more general fashion as \autoref{eq:PowerLaw_1}, where $A$ corresponds to the normalization constant and can be found using \autoref{eq:NormConst} with $\gamma = 1 - \alpha$.\\

\begingroup
\Large
\begin{equation}
 \frac{\textnormal{dN}}{\textnormal{dX}}~~=~~\textnormal{A}~\textnormal{X}^\alpha~~~~~~~;~~~~~~~x_1 ~~\leq ~~\textnormal{X}~~ \leq ~~\textnormal{x}_2
 \label{eq:PowerLaw_1}
\end{equation}
\endgroup\\

\begingroup
\Large
\begin{equation}
 \textnormal{A}~~=~~\int_{\textnormal{x}_1}^{\textnormal{x}_2} X^{-\alpha} \textnormal{dX}~~=~~\frac{\textnormal{x}_2^{1 - \alpha} - \textnormal{x}_1^{1 - \alpha}}{1 - \alpha}~~=~~\frac{\textnormal{x}_2^{\gamma} - \textnormal{x}_1^{\gamma}}{\gamma} 
 \label{eq:NormConst}
\end{equation}
\endgroup\\

Furthermore, we can define the cumulative distribution function (CDF) $F(X)$, which will give us all the accumulated probability less than or equal to $X$. It is widely used to determine the probability of an observation being greater than a certain value, or between two values. The CDF will be of great importance in \autoref{sec:MCSec} where the randomly distributed variable is obtained making use of it to generate the real variable. The mathematical form of the CDF is given in \autoref{eq:CDF} where the upper limit in the integral $x$ here refers to a value between the upper limit ($x_1$) and lower limit ($x_2$) over which one wants to generate the distribution.\\ 

\begingroup
\Large
\begin{equation}
 \textnormal{F}(\textnormal{x})~~=~~\textnormal{A}^{-1} \int_{\textnormal{x}_1}^{\textnormal{x}} \textnormal{t}^{-\alpha} \textnormal{dt}~~=~~\frac{\textnormal{x}^{\gamma} - \textnormal{x}_1^{\gamma}}{\textnormal{x}_2^{\gamma} - \textnormal{x}_1^{\gamma}}
 \label{eq:CDF}
\end{equation}
\endgroup\\

Subsequently, if the random variable is distributed uniformly between $0$ and $1$, one can generate the real variable by inverting the CDF shown in \autoref{eq:CDF} which leads to \autoref{eq:RandomVar}. As expected, if one evaluates the last equation in $y = 0$ and $y = 1$ which are the extreme values of the random variable, the result is $x = x_1$ and $x = x_2$ respectively in the real variable.\\

\begingroup
\Large
\begin{equation}
  \begin{align*}
    \textnormal{y} & ~~=~~  \textnormal{F}(\textnormal{x}) ~~=~~ \frac{\textnormal{x}^{\gamma} - \textnormal{x}_1^{\gamma}}{\textnormal{x}_2^{\gamma} - \textnormal{x}_1^{\gamma}} \\[10pt]
    \textnormal{x} & ~~=~~  (y~ (~\textnormal{x}_2^{\gamma} - \textnormal{x}_1^{\gamma}~) + \textnormal{x}_1^{\gamma})^{1/\gamma}
    \label{eq:RandomVar}
  \end{align*}
\end{equation}
\endgroup\\

The planetary mass-period distribution, and the stellar mass distribution will be generated following the simple power law explained above with a Monte-Carlo process in \autoref{sec:MCSec}. Although a different method was also explored for the planetary mass-period distribution due to a possible weakly dependence in both parameters (\cauthor{1538-3881-134-5-2061}\citeyear{1538-3881-134-5-2061}, \cauthor{2002ApJ...568L.113Z}\citeyear{2002ApJ...568L.113Z}), we decided to kept the power law method to model them as it is also widely used and studied in literature (\cauthor{2010EAS....41..107N}\citeyear{2010EAS....41..107N}, \cauthor{2008PASP..120..531C}\citeyear{2008PASP..120..531C}, \cauthor{2006ApJ...646..505B}\citeyear{2006ApJ...646..505B}).  

\section{Exoplanets: Period-Mass Distributions}

In order to draw a reliable distribution sample of period and mass for exoplanets, we used two different approaches. The first approach uses the $\beta$-distribution because there exits a weakly correlation between the period and mass of an exoplanet as shown by \cauthor{2002ApJ...568L.113Z}\citeyear{2002ApJ...568L.113Z} which makes the distribution analysis not suitable to be addressed by two independent power laws that describe the joint period-mass distribution. Alternatively, one can assume that as the correlation is weak, then each variable can be treated as independent and the distributions may be generated using single power laws as it is also widely explored by other authors (\cauthor{2010EAS....41..107N}\citeyear{2010EAS....41..107N}). Therefore, as there exists two different forms to address the generation of these distributions, both ways were explored and implemented in this work.\\

In \autoref{subsec:BetaDist} and \autoref{subsec:SPLDist}, the method is widely explained taking into account the different observations and arguments of the former authors. 

\subsection{\texorpdfstring{$\beta$-}Distribution}\label{subsec:BetaDist}

Using a data set of 66 exoplanets \cauthor{2002ApJ...568L.113Z}\citeyear{2002ApJ...568L.113Z} suggested a possible correlation between the mass and period. Subsequently \cauthor{1538-3881-134-5-2061}\citeyear{1538-3881-134-5-2061} using a data set of 233 exoplanets supported this idea measuring a positive correlation coefficient of $0.1762$. As a result of the positive correlation, describing the distribution as two independent power laws it is not correct, and a new coupled positively correlated function is needed to describe the problem. However, generating this type of distributions needs $\beta$-distributed random variables which was not provided until \cauthor{2004CMS...46.397M}\citeyear{2004CMS...46.397M} work.\\

The probability distribution function (pdf) on a finite interval (c,d), $-\infty < \textnormal{c} < \textnormal{d} < \infty $, indexed by two positive parameters $\alpha$ and $\beta$ is given by \autoref{eq:BetaPDF}, where $\textnormal{B}(\alpha, \beta)$ denotes the beta function and can be computed using \autoref{eq:BetaExplicit}.\\

\begingroup
\Large
\begin{equation}
 \textnormal{f}_\beta(\textnormal{x} | \alpha, \beta)~~=~~\frac{1}{\textnormal{B}(\alpha, \beta)}\frac{(\textnormal{x}-\textnormal{c})^{\alpha -1} (\textnormal{d}-\textnormal{x})^{\beta -1}}{(\textnormal{d}-\textnormal{c})^{\alpha + \beta - 1}}~~~~~~~;~~~~~~~\textnormal{c}~~ \leq ~~\textnormal{x} ~~\leq ~~\textnormal{d},~~~\alpha>0,~~~\beta>0
 \label{eq:BetaPDF}
\end{equation}
\endgroup

\begingroup
\Large
\begin{equation}
 \textnormal{B}(\alpha, \beta)~~=~~\int_0^1 \textnormal{t}^{\alpha -1}(1-\textnormal{t})^{\beta -1}\textnormal{dt}~~=~~\frac{\Gamma(\alpha)\Gamma(\beta)}{\Gamma(\alpha + \beta)}
 \label{eq:BetaExplicit}
\end{equation}
\endgroup\\

Using the correct transformation, the pdf can be written in terms of a normal distributed variable to obtain the standard $\beta$-distribution as shown in \autoref{eq:BetaStandard} which is a useful form to implement the algorithm provided in \cauthor{2004CMS...46.397M}\citeyear{2004CMS...46.397M}.\\   

\begingroup
\Large
\begin{equation}
  \textnormal{f}(\textnormal{y} \mid \alpha, \beta)~~=~~\frac{1}{\textnormal{B}(\alpha, \beta)}\textnormal{y}^{\alpha -1}(1-\textnormal{y})^{\beta-1}~~~~~~~;~~~~~~~\textnormal{0} ~~\leq ~~\textnormal{y} ~~\leq ~~\textnormal{1},
 \label{eq:BetaStandard}
\end{equation}
\endgroup\\

The final distribution for mass and period can be obtained through \autoref{eq:FinalDist}, where $(\hat{\alpha}_m, \hat{\beta}_m) = (0.6524, 5.9070)$ and $(\hat{\alpha}_p, \hat{\beta}_p) = (0.3697, 3.8445)$ as a result of applying a Maximum-Likelihood Method to their observational data. The normalization constants in both cases are given by $A_1 = 115.5$ and $A_2 = 11650$, corresponding to the area below the observed histogram distribution for each one of the parameters. 

\begingroup
\Large
\begin{equation}
  \begin{align*}
    \textnormal{f}_\beta^{\textnormal{M}} & = \textnormal{A}_1\textnormal{f}_\beta(\textnormal{m}|\hat{\alpha}_m, \hat{\beta}_m) \\[10pt]
    \textnormal{f}_\beta^{\textnormal{P}} & = \textnormal{A}_2\textnormal{f}_\beta(\textnormal{p}|\hat{\alpha}_p, \hat{\beta}_p)
    \label{eq:FinalDist}
  \end{align*}
\end{equation}
\endgroup\\

In short, the mass and period distributions can be then generated using \autoref{eq:BetaStandard} and \autoref{eq:FinalDist} through a Monte-Carlo process. These equations can be read as the probability of a planet to be in a mass range $[M, M + dM]$ and a period range $[P, P + dP]$. The actual upper and lower limits in mass, and period are given by the data set used to derive the normalization constant and the index of the $\beta$-distribution. Thus, we can generate samples in mass-period ranges of $0.008 < M(M_j) < 26.7$ and $0.8079 < P(days) < 6776.1$. The application of this model to our current problem is shown and discussed in \autoref{sec:MCSec}.


\subsection{Single Power-Law}\label{subsec:SPLDist}

As discussed in \autoref{sec:PowerLawSec} and \autoref{subsec:BetaDist}, the planetary mass and period are weakly correlated, so one can ignore that and addressed the problem as independent single power laws. In the past, this has been studied by (\cauthor{2008PASP..120..531C}\citeyear{2008PASP..120..531C}, \cauthor{2006ApJ...646..505B}\citeyear{2006ApJ...646..505B}) considering the distributions of semimajor axis and planet mass of known radial velocity planets. However, in recent studies, \cauthor{2010EAS....41..107N}\citeyear{2010EAS....41..107N} noted that due to a decrease in sensitivity of the radial velocity method with orbital distance the exponent of the distribution must be modified. The single power law distributions in mass, semimajor axis and period are shown in \autoref{eq:MassPeriodDist}.\\

\begingroup
\Large
\begin{equation}
  \begin{align*}
    \frac{\textnormal{dN}}{\textnormal{dm}}~~&\propto~~\textnormal{m}^{-1.16} \\[10pt]
    \frac{\textnormal{dN}}{\textnormal{da}}~~&\propto~~\textnormal{a}^{-0.61} \\[10pt]
    \frac{\textnormal{dN}}{\textnormal{dP}}~~&\propto~~\textnormal{P}^{-0.74}
    \label{eq:MassPeriodDist}    
 \end{align*}
\end{equation}
\endgroup\\

In the same way as stated before, we can interpret the former equation as the number of planets expected to be contained in a mass range $m_1 < m < m_2$, a semimajor axis range $a_1 < a < a_2$, and orbital period $p_1 < p < p_2$. Whereas in the case of $\beta$-distributions, the mass and period cover a wide range, here the mass is reduced to a range of $0.5 < M(M_j) < 13$ and an upper cut-off at $75 AU$ which leads to an upper limit of $\sim 650$yr in period allowing to study wider planetary orbits.    

\section{Stellar Mass Distribution}

Apart from modeling the planetary mass and period, we aimed to obtain in the same fashion the stellar mass distribution. This is known as the initial mass function (IMF) and it is still a wide open question in current Astrophysics. There exists different power laws which try to describe the number of stars expected to lie in a given mass range. In this work, we decided to test two different forms of the IMF namely the Salpeter power law proposed by Edwin Salpeter in $1955$ (\cauthor{1955ApJ...121..161S}\citeyear{1955ApJ...121..161S}) and the Kroupa power law proposed by Pavel Kroupa in $2001$ (\cauthor{2001MNRAS.322..231K}\citeyear{2001MNRAS.322..231K}). The main difference between these two formulations resides on the value that each exponent can take according to each mass range in which one could be interested in. The main goal in using these power laws is to faithfully reproduce the actual observed IMF distribution of stars in a given mass range using the Monte-Carlo process technique. In \autoref{subsec:Salpeter} and \autoref{subsec:Kroupa} a brief introduction of the main features for both power laws is given.    

\subsection{Salpeter Power-Law}\label{subsec:Salpeter}

In 1955, Edwin Salpeter used the observed luminosity function for main-sequence stars in the solar neighborhood assuming that stars off the main-sequence have already burnt up $~10\%$ of their hydrogen mass, and also that stars in the solar neighborhood have been created at a uniform rate for the last five billion years to compute the rate of star creation as a function of stellar mass, and the number of stars in each mass range \cauthor{1955ApJ...121..161S}\citeyear{1955ApJ...121..161S}. Having said that, he found the power law describing the IMF to follow \autoref{eq:Salpeter_1}, in which $\xi_0$ is a constant related to the local stellar density and $\alpha = 2.35$. The former equation gives us the number of stars expected to be in a mass range $[M$, $M + dM]$.\\ 

\begingroup
\Large
\begin{equation}
  \xi(\textnormal{m})\Delta \textnormal{m} = \xi_0 \left(\frac{\textnormal{m}}{\textnormal{M}_\odot}\right)^{-2.35} \left(\frac{\Delta \textnormal{m}}{\textnormal{M}_\odot}\right)
 \label{eq:Salpeter_1}
\end{equation}
\endgroup\\

As we are interested in using our own mass range, and just make use of the exponent to draw a mass distribution we can rewrite \autoref{eq:Salpeter_1} into \autoref{eq:Salpeter_2}, and later apply all the steps listed in \autoref{sec:PowerLawSec} to later make use of the Monte-Carlo process and obtain our sample of modeled stars. The proportionality constant can be found once the total number of stars in a mass range $m_1 < m < m_2$ is known, through \autoref{eq:NormConst}.

\begingroup
\Large
\begin{equation}
  \frac{\textnormal{dN}}{\textnormal{dm}} \propto m^{-2.35}
 \label{eq:Salpeter_2}
\end{equation}
\endgroup\\

\subsection{Kroupa Power-Law}\label{subsec:Kroupa}

On the other hand, in $2001$, a different formulation was proposed by Pavel Kroupa in which the main feature is a change in the slope (power-law index) near to $0.08M_\odot$ and $0.5M_\odot$ \cauthor{2001MNRAS.322..231K}\citeyear{2001MNRAS.322..231K}. In other words, the number of stars expected in a given mass range has different values for the power-law exponent in contrast to Salpeter's law which has only one index. The general form is given by \autoref{eq:Kroupa_1}. One interesting feature of this power law is that $50\%$ of the data generated falls into the mass range $0.01 \leq \frac{m}{M_\odot} \leq 1.0$, and $50\%$ falls into $1.0 \leq \frac{m}{M_\odot} \leq 50.0$. 

\begingroup
\Large
\begin{equation}
    \xi(\textnormal{m}) \propto m^{-\alpha_0} = 
    \begin{cases}
     \alpha_0 = +0.3 \pm 0.7,& \text{if } ~~ 0.01 \leq \frac{m}{M_\odot} \leq 0.08\\
     \alpha_0 = +1.3 \pm 0.5,& \text{if } ~~ 0.08 \leq \frac{m}{M_\odot} \leq 0.50\\
     \alpha_0 = +2.3 \pm 0.3,& \text{if } ~~ 0.50 \leq \frac{m}{M_\odot} \leq 1.00\\
     \alpha_0 = +3.3 \pm 0.7,& \text{if } ~~ 1.00 \leq \frac{m}{M_\odot}
    \end{cases}
\label{eq:Kroupa_1}
\end{equation}
\endgroup\\

The IMF generation will be addressed in the same fashion as explained above for the Salpeter's power law, where a Monte-Carlo process will be used and the normalization constant will be set to the total number of stars in a mass range $m_1 < m < m_2$. 

\section{Monte-Carlo simulations}\label{sec:MCSec}

\begin{figure}[!ht]
\centering
  \subfloat{\includegraphics[width = 12cm, height = 9cm]{./Graficos/Capitulo_2/2_Exop_distributions/Period_mass_distribution_2.png}} 
\caption{\scriptsize{Something!}}
\label{fig:PeriodMass_Beta}
\end{figure}

\begin{figure}[!ht]
\centering
  \subfloat{\includegraphics[width = 12cm, height = 9cm]{./Graficos/Capitulo_2/2_Exop_distributions/Period_major_distribution_2.png}} 
\caption{\scriptsize{Something!}}
\label{fig:PeriodMajor_Beta}
\end{figure}

\begin{figure}[!ht]
\centering
  \subfloat{\includegraphics[width = 12cm, height = 9cm]{./Graficos/Capitulo_2/2_Exop_distributions/Mass_major_distribution_2.png}} 
\caption{\scriptsize{Something!}}
\label{fig:MassMajor_Beta}
\end{figure}

\begin{figure}[!ht]
\centering
  \subfloat{\includegraphics[width = 18.5cm, height = 15.5cm, scale = 1.0, angle = 90]{./Graficos/Capitulo_2/2_Exop_distributions/Mass_distribution_Nielsen.png}} 
\caption{\scriptsize{Something!}}
\label{fig:Mass_Nielsen}
\end{figure}

\begin{figure}[!ht]
\centering
  \subfloat{\includegraphics[width = 18.5cm, height = 15.5cm, scale = 1.0, angle = 90]{./Graficos/Capitulo_2/2_Exop_distributions/Period_distribution_Nielsen.png}} 
\caption{\scriptsize{Something!}}
\label{fig:Mass_Nielsen}
\end{figure}

\begin{figure}[!ht]
\centering
  \subfloat{\includegraphics[width = 12cm, height = 9cm]{./Graficos/Capitulo_2/2_Exop_distributions/Period_mass_distribution.png}} 
\caption{\scriptsize{Something!}}
\label{fig:PeriodMass_Nielsen}
\end{figure}

\begin{figure}[!ht]
\centering
  \subfloat{\includegraphics[width = 12cm, height = 9cm]{./Graficos/Capitulo_2/2_Exop_distributions/Period_major_distribution.png}} 
\caption{\scriptsize{Something!}}
\label{fig:PeriodMajor_Nielsen}
\end{figure}

\begin{figure}[!ht]
\centering
  \subfloat{\includegraphics[width = 12cm, height = 9cm]{./Graficos/Capitulo_2/2_Exop_distributions/Mass_major_distribution.png}} 
\caption{\scriptsize{Something!}}
\label{fig:MassMajor_Nielsen}
\end{figure}


%============================================================================================================================================================

%============================================================================================================================================================
\section{Probability of Transit Detection}

%Star with a planet, Gaia detectability

One of the main goals in this work is to constrain the probability of detecting an exo-ring transit around young stars using \textit{Gaia} observations. We should start thinking about the possible factors which could affect the most their detectability such as the geometry of the transit, the chance to observe a star with planets around it, the probability of detecting any feature with \textit{Gaia}'s cadence, the time it takes to form a ring around a planet and how long it lasts, or the probability of a given planet to have its Hill sphere filled with some material which could possible form rings. A few of this probabilities are hard to compute, basically because the only knowledge we have is provided through observations of our own solar system as could be the rings lifetime. However, we can make our best guess and provide at least a lower boundary of the probability of transit detection, and lately obtain the number of planets one would expect to observe given some survey features.\\

First, we decided to constrain our detectability prediction as a product of five independent probabilities  as shown in \autoref{eq:Probabilities}, where $P_1$ corresponds to the probability of a given star to have a planet, $P_2$ gives the probability of a planet to have its Hill sphere filled with material that would coalesce and form rings, $P_3$ constrains the probability of observing exoplanetary rings transiting in front of their parent star given an observer in the universe, and $P_4$ the probability of observing at least one transit with \textit{Gaia} in all the mission lifetime. Apart from these four probabilities, we included another one to account for the rings lifetime but it was addressed separately to study how this could affect the overall outcome and is explained in \autoref{RingsSec}.\\

\begingroup
\Large
\begin{equation}
 \textnormal{P}_{\textnormal{transit}} = \textnormal{P}_1 * \textnormal{P}_2 * \textnormal{P}_3 * \textnormal{P}_4
 \label{eq:Probabilities}
\end{equation}
\endgroup

On top of that, we can start constructing each probability in terms of their main variables. Firstly, the probability of a star having a planet was set to a value of $P_1 = 0.17$ which means that a star has on average a $17\%$ chance of hosting a planet. In \cauthor{2012Natur.481..167C}\citeyear{2012Natur.481..167C},  


\subsection{Geometry of the Transit}
\subsection{Hill Sphere}
\subsection{Rings Lifetime Constrain}\label{subsec:RingsSec}

%============================================================================================================================================================


%============================================================================================================================================================
\subsection{Analytic Form}

