%*****************************************
\chapter{ \textbf{Theoretical framework} }\label{ch:Theoretical}
%*****************************************
\vspace{0.5cm} 

\section{Introduction}

During the last years, the exo-planetary astrophysics field has been getting stronger and stronger. Searching for new worlds revolving around other stars have imposed important challenges in current science such as planetary formation models, observational and instrumentation challenges, for example. With the arrival of new instruments, each time more powerful, Astronomers are able to study and characterize these distant worlds and lately compare them with our solar system. In despite of it, we still have a long way to go in studying and understanding these interesting objects.\\

This project is mainly focused on studying exo-planets with rings around young stars. At the moment, there is a lot of debate on whether or not, we have observed some features in light curves which could be explained by transiting exo-rings in front of a parent star. On top of that, it is well-known that rings are not an exception in our solar system, where we can observe majestic structures as Saturn's rings or modest ones as the other gaseous-planets. As there is no clear consensus at what time exactly during the stellar life and planetary formation this objects could be formed, we aimed to enhance our chance of detecting these structures while studying young stars. These particular stellar population is expected to be forming planets at early stages which makes them good candidates in our search. The targeted field is known as Sco-Cen, a really young OB stellar stellar association at a distance of $100-100\mathrm{pc}$ from the sun. Sco-Cen is composed of (number of star, or any other information) and it is located between the constellations of Scorpius, Centaurus and Crux in the southern hemisphere.\\ 

In addition, all the characterization of astrophysical sources is mainly dependent on the relative distance to the observer. Therefore, aiming for excellent measurements of distance is essential to properly address our study. A few years ago in (year), the \textit{Gaia} mission was launched to measure with high precision the parallax of stars using their proper motion. As \textit{Gaia} samples the whole celestial sphere, and we need as much as possible accurate measurements for stars in Sco-Cen, we decided to use this mission.\\

In this chapter, a brief introduction to planet-formation and exo-rings is provided. Also, we describe the most relevant features of the \textit{Gaia} mission, and a general description of the Sco-Cen OB association. 

\section{Exo-Rings}
\section{Gaia mission}
\section{Sco-Cen}


