%*****************************************
\chapter{\textbf{Conclusions and Discussion} }\label{ch: Results}

%*****************************************
\vspace{0.5cm} 
%============================================================================================================================================================

\section{Introduction}

In this chapter, we address a brief discussion on the most relevant points of our work. As can be noticed, our work focuses on studying the occurrence rate of stellar rings following Monte-Carlo and analytic approaches to solve a proposed probability distribution for these events. On the other hand, we decided to start paving the road for future studies by analyzing a particular region of the sky known as ScoCen which enhances our chance to look for exoplanetary rings because it contains a huge sample of young stars. As \textit{Gaia} DR2 provides utterly exquisite data with parallaxes, proper motions, and positions measurement allowing us to constrain a final sample of stars in the ScoCen subgroups to be properly cross-matched them with the \textit{SuperWASP} database. The former database provides light curves obtained after three years of observations particularly useful for the \textit{Upper Scorpius} and \textit{Upper Centaurus-Lupus} regions which lie in a region of the sky pretty well sampled by \textit{SuperWASP}. However, for \textit{Lower Centaurus-Crux} we report a small fraction of light curves due to the lack of data close to the Milky Way mid-plane. Different candidates are proposed and presented as a final sample to be followed up in other surveys and future works.  

\section{Results and Discussion}

We have proposed an exoplanetary ringed system probability with five independent components (see \autoref{eq:Probabilities}). The first component $\textnormal{P}_1$ gives the chance of a star in the night sky to have planets orbiting around. This was set to $17\%$ as reported in literature (\cauthor{2012Natur.481..167C} \citeyear{2012Natur.481..167C}) from Hot-Jupiter distribution studies. On the other hand, we proposed the second probability $\textnormal{P}_2$ corresponding to the chance of a given planet to have material orbiting around inside the Hill sphere to be $100\%$ in the most optimistic case because we expect to study really young stars in which material coalescing it is likely to be taking place. The third probability $\textnormal{P}_3$ gives us insights on the geometry of the transit. In this case, we considered grazing and full transits assuming that the planet radius is small enough compared to the Hill sphere, and also assuming transits can be seen for an observer on Earth with a different inclination that in overall will be distributed randomly. This lead to \autoref{eq:ProbTransit_4} in which the Hill sphere plays an important role in our calculations. As the future plan consists of using the photometric data for a given star in \textit{Gaia} database after the nominal five-year mission, we decided to include a fourth probability $\textnormal{P}_4$ that accounts for observing at least one exoplanetary ring transit in the averaged $~70$ data points per star observed by \textit{Gaia}. Last but not least, the fifth probability $\textnormal{P}_5$ was introduced to account for the fact that exoplanetary rings timescales are not really well understood and the only examples we have, corresponding to our own solar system. There is no agreement on whether they form at the same time of the star-planet formation process or in between. As this is one of the trickiest value, we proposed a ratio between values reported in literature for studies of Saturn, Jupiter, Uranus and Neptune rings (\cauthor{1984prin.conf..641H} \citeyear{1984prin.conf..641H}; \cauthor{1994P&SS...42.1139C} \citeyear{1994P&SS...42.1139C}; \cauthor{2009sfch.book..537C} \citeyear{2009sfch.book..537C}; \cauthor{2013pss3.book..309T} \citeyear{2013pss3.book..309T}) in which an average value is close to $~10^6\textnormal{Myr}$, and the stellar age which was set to a fixed number of $20\textnormal{Myr}$ after comparing with isochrone and stellar-track models of the ScoCen region.\\

Exoplanetary ring transits are more likely to be detected around low-mass stars as presented in \autoref{fig:MonteKroupaNielsen}, where the occurrence rate is higher for this type of stars independent of the mass-period exoplanet pairs selected. This can also be seen in the $0.4\textnormal{M}_\odot$ binned figure that gives us the expected number of observations. This is one of the main reasons we decided to study a young region as ScoCen because it enhances our chance to obtain light curves with possible features related to exoplanetary ring transits. The ring timescale was observed to affect the outcome and highly control the occurrence/number of transits expected. The younger the rings compared to the stellar age, the lower the chance to observe them. We tested different timescales in which the lowest values for the occurrence correspond to $~10^5\textnormal{Myr}$ and the highest for $~10^7\textnormal{Myr}$ (see \autoref{fig:Rings_Prob}). If we considered the case in which the fifth probability is not included, then we have an upper limit for the occurrence values. Knowing really well how rings evolve, when they are formed, and how long do they survive after the final planetary system set up is vital in order to faithfully reproduce and study the probability distribution of these systems. Future studies and observations may help to study these probabilities individually and constraint more precisely the occurrence of exoplanetary rings. Nevertheless, it seems that there is a high chance to observe them at least in a broad way of speaking, because low-mass stars are really abundant, although these stars are the hardest to follow up with spectroscopic and photometric surveys.\\

As \textit{Gaia} DR2 represents the most precise survey at positions and parallaxes in the sky, we decided to use this data to pre-select stars in the ScoCen region to search for young low-mass stars motivated by the probabilistic analysis explained above. A pre-selection of stars in the \textit{Lower Centaurus-Crux} (LCC), \textit{Upper Centaurus-Lupus} (UCL), and \textit{Upper Scorpius} (US) subgroups was proposed following works of (\cauthor{Blaauw46} \citeyear{Blaauw46}; \cauthor{1999AJ....117..354D} \citeyear{1999AJ....117..354D}). \textit{Gaia} DR2 contains a vast number of stars with spurious parallaxes and photometry, thus, we performed a cleaning process of our samples before analyzing the data as proposed by \cauthor{2018arXiv180409366L} (\citeyear{2018arXiv180409366L}). This led to a final of $3729$, $3309$ and $3432$ in the LCC, UCL, and US subgroups respectively. Stellar-track and isochrone models were computed using the \textit{MIST}-interface which implements MESA-code to guarantee young low-mass stars in our sample. The lowest limit of the model in stellar mass is $0.1\textnormal{M}_\odot$, however, as it is shown in \autoref{fig:DR2_Tracks}, we have a vast number of stars well below this limit. This shows how powerful \textit{Gaia} is to sample the dwarf stellar population. The final sample of stars was selected to be between $5$ to $60\textnormal{Myr}$ and then compared to the \textit{SuperWASP} database. In doing so, we obtain a sample of $113$, $654$, and $718$ light curves for the LCC, UCL, and US subgroups respectively. Most of the stars seem to follow a regular behavior during the three years of observations of the \textit{SuperWASP} project. However, some stars give the impression of variation or simply features like those shown for J1407. Thus, we report a number of candidates to be follow up in future studies as presented in \autoref{tab:Candidates_Final}.\\

The exoplanetary ring candidates proposed in \autoref{tab:Candidates_Final} opens the possibility of studying in depth the well-known J1407 system in comparison to other sources in the same region of the sky. We did not perform any light curve averaging per night or light curve fitting, but this is essential to rule out spurious candidates and set a more reliable final sample which could be followed up by future observations to confirm this fascinating transiting objects. As \textit{Gaia} aims to revolutionize our current understanding of stars in the Milky Way, and allows us to go down as never before in magnitude opening up a new window on low-mass and substellar objects, it is quite important to plan future exoplanetary transit surveys capable of observing rings constraint through this new unprecedentedly samples.\\

Aside from the exoplanetary analysis, we used \textit{Gaia} DR2 data to explore its current capabilities in the ScoCen region. Motivated by \cauthor{Skrutskie06} (\citeyear{Skrutskie06}) and \cauthor{Zari17} (\citeyear{Zari17}) who studied the Orion region using the pre-main sequence stars, we proposed to restrict our \textit{Gaia} DR2 query only in space i.e. in parallax/distance and galactic coordinates, but forgetting about any other known evidence on the proper motions derived in the past for the ScoCen region. In \autoref{fig:ScoOB_premain}, the original three boxes as defined by \cauthor{Blaauw46} \citeyear{Blaauw46} were expanded to include a larger area and to get rid of the common boxes, while the parallax was set to lie in a range of $5$ to $12$ mas to include possible new candidates in the total sample. if we select only the pre-main sequence stars from the Hertzsprung-Russel diagram by performing a polygon selection (this can also be done by using the isochrone approach to be more precise) in which we can guarantee that we have low-mass stars $< 2.0 \textnormal{M}_\odot$ and young stars $> 15 \textnormal{Myr}$, the distribution in galactic coordinates can be drawn. \autoref{fig:ScoOB_premain} right-panel shows how one can draw the general trend and provide a fast understanding of the whole association by studying the agglomerations in space. This method, combined with \textit{Gaia} data provides an easy way to tackle down membership studies in associations before applying more rigorous methods involving kinematics and dynamics enhancing the chance to obtain a member of the actual association. The \textit{Upper Scorpius} region is well drawn, however, the \textit{Upper Centaurus-Lupus} and \textit{Lower Centaurus-Crux} pops out naturally and one can observe a real trend in the distance and spatial distribution. IC $2602$ is also present in the lowest right-hand side of the figure. Outside of the regular Upper Scorpius region, a notorious over-density is present. This was spotted before by \cauthor{1999AJ....117..354D} (\citeyear{1999AJ....117..354D}) and also \cauthor{Mamajek2016} (\citeyear{Mamajek2016}) using data from\textit{Gaia} DR1. Therefore, it is not a surprise that we can retrieve a clearer view as we have access to better and larger samples of data with \textit{Gaia} DR2. Nevertheless, as stated before, this not only shows how powerful \textit{Gaia} can be but also how useful this method is to easily detect possible new members.\\ 

Summarizing, the occurrence of exoplanetary rings is quite likely to happen around young low-mass stars (below $2.0\textnormal{M}_\odot$ in mass and around $15\textnormal{Myr}$ in age). On the other hand, \textit{Gaia} provides an excellent and exquisite dataset allowing a deeper and in detail study of the low-mass stellar population which in different surveys is not given due to photometric and spectroscopic limitations. \textit{Gaia} provides exquisite data that can be easily matched with other databases to perform amazing scientific research as the one describes in this project.      
