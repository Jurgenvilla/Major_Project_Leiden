%*******************************************************
% Abstract
%*******************************************************
%\renewcommand{\abstractname}{Abstract}
\pdfbookmark[1]{Abstract}{Abstract}
\begingroup
\let\clearpage\relax
\let\cleardoublepage\relax
\let\cleardoublepage\relax

% \chapter*{Abstract}
\chapter*{Abstract}


\vfill
In this work, we present a probabilistic analysis of the occurrence rate of exoplanetary rings. Following a Monte-Carlo and analytic approach to reproduce the stellar mass IMF and the exoplanetary mass-and period-distributions, we proposed low-mass stars below $2.0 \textnormal{M}_\odot$ as those candidates with the highest chance to harbor planets orbiting with exoplanetary rings (already formed or in the formation process). The expected number of detections lies between $100$ and $10$ for a stellar mass range between $0.1 \textnormal{M}_\odot$ and $2.0 \textnormal{M}_\odot$, out of a total number of $10\,000$ stars and exoplanets period-mass pairs simulated. Using \textcolor{halfgray}{\textit{Gaia} DR2} data we performed an analysis of the well-known OB association ScoCen (also known as Sco OB2) subdividing the region in three subgroups, \textcolor{halfgray}{\textit{Upper Scorpius}}, \textcolor{halfgray}{\textit{Upper Centaurus-Lupus}}, and \textcolor{halfgray}{\textit{Lower Centaurus-Crux}}. For each subgroup, its respective Hertzsprung-Russell was created to pre-select stars between two isochrone models of $5 \textnormal{Myr}$ and $60 \textnormal{Myr}$, where the youngest isochrones were corrected by extinction values reported in the literature. Stellar tracks where also computed probing that our final sample contains stars below $2 \textnormal{M}_\odot$ to values down to $0.1 \textnormal{M}_\odot$ which corresponds to the lower limits of the isochrone-and stellar track-models. However, we can go a bit further down this limit in mass with the new \textcolor{halfgray}{\textit{Gaia} DR2} data set. The final sample contains $3\,729$, $3\,309$, and $3\,432$ sources in each one of the \textcolor{halfgray}{LCC}, \textcolor{halfgray}{UCL} and \textcolor{halfgray}{US} subgroups. This final sample of stars is cross-matched to the \textcolor{halfgray}{\textit{SuperWASP}} database in order to retrieve light curves for the selected stars to be studied, motivated by the light curve of J1407. This cross-matching process let us with $113$, $654$, and $718$ light curves per subgroup but only a final sample of $\sim20$ candidates is proposed after inspecting the light curves and comparing to the actual light curve of J1407 searching for possible exoplanetary ringed systems or variable star behavior. Additionally, we redefined the original limits of ScoCen in galactic coordinates to include a larger sky area in a range of $285^\circ\leq\ell\leq360^\circ$ and $-10^\circ\leq b\leq+32^\circ$ and a parallax range of $5$ to $12$ mas and performed a selection of the young population which led to the natural overall trend and some outstanding over-densities close to the original boxes. This method allows the study of membership in associations relatively quick as a first step to lately applied a kinematic method and increase the chance of studying a large sample of actual stars belonging to the association.\medskip \\

Keywords: \textit{planets and satellites: rings --- parallaxes --- techniques: photometric --- open clusters and associations: individual (Sco-Cen)}.

\endgroup			

\vfill